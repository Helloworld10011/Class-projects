\documentclass[]{article}
\usepackage{lmodern}
\usepackage{amssymb,amsmath}
\usepackage{ifxetex,ifluatex}
\usepackage{fixltx2e} % provides \textsubscript
\ifnum 0\ifxetex 1\fi\ifluatex 1\fi=0 % if pdftex
  \usepackage[T1]{fontenc}
  \usepackage[utf8]{inputenc}
\else % if luatex or xelatex
  \ifxetex
    \usepackage{mathspec}
  \else
    \usepackage{fontspec}
  \fi
  \defaultfontfeatures{Ligatures=TeX,Scale=MatchLowercase}
\fi
% use upquote if available, for straight quotes in verbatim environments
\IfFileExists{upquote.sty}{\usepackage{upquote}}{}
% use microtype if available
\IfFileExists{microtype.sty}{%
\usepackage{microtype}
\UseMicrotypeSet[protrusion]{basicmath} % disable protrusion for tt fonts
}{}
\usepackage[margin=1in]{geometry}
\usepackage{hyperref}
\hypersetup{unicode=true,
            pdftitle={Causal Inference: HW1},
            pdfauthor={Mahdi Sabbaghi (95109123)},
            pdfborder={0 0 0},
            breaklinks=true}
\urlstyle{same}  % don't use monospace font for urls
\usepackage{color}
\usepackage{fancyvrb}
\newcommand{\VerbBar}{|}
\newcommand{\VERB}{\Verb[commandchars=\\\{\}]}
\DefineVerbatimEnvironment{Highlighting}{Verbatim}{commandchars=\\\{\}}
% Add ',fontsize=\small' for more characters per line
\usepackage{framed}
\definecolor{shadecolor}{RGB}{248,248,248}
\newenvironment{Shaded}{\begin{snugshade}}{\end{snugshade}}
\newcommand{\AlertTok}[1]{\textcolor[rgb]{0.94,0.16,0.16}{#1}}
\newcommand{\AnnotationTok}[1]{\textcolor[rgb]{0.56,0.35,0.01}{\textbf{\textit{#1}}}}
\newcommand{\AttributeTok}[1]{\textcolor[rgb]{0.77,0.63,0.00}{#1}}
\newcommand{\BaseNTok}[1]{\textcolor[rgb]{0.00,0.00,0.81}{#1}}
\newcommand{\BuiltInTok}[1]{#1}
\newcommand{\CharTok}[1]{\textcolor[rgb]{0.31,0.60,0.02}{#1}}
\newcommand{\CommentTok}[1]{\textcolor[rgb]{0.56,0.35,0.01}{\textit{#1}}}
\newcommand{\CommentVarTok}[1]{\textcolor[rgb]{0.56,0.35,0.01}{\textbf{\textit{#1}}}}
\newcommand{\ConstantTok}[1]{\textcolor[rgb]{0.00,0.00,0.00}{#1}}
\newcommand{\ControlFlowTok}[1]{\textcolor[rgb]{0.13,0.29,0.53}{\textbf{#1}}}
\newcommand{\DataTypeTok}[1]{\textcolor[rgb]{0.13,0.29,0.53}{#1}}
\newcommand{\DecValTok}[1]{\textcolor[rgb]{0.00,0.00,0.81}{#1}}
\newcommand{\DocumentationTok}[1]{\textcolor[rgb]{0.56,0.35,0.01}{\textbf{\textit{#1}}}}
\newcommand{\ErrorTok}[1]{\textcolor[rgb]{0.64,0.00,0.00}{\textbf{#1}}}
\newcommand{\ExtensionTok}[1]{#1}
\newcommand{\FloatTok}[1]{\textcolor[rgb]{0.00,0.00,0.81}{#1}}
\newcommand{\FunctionTok}[1]{\textcolor[rgb]{0.00,0.00,0.00}{#1}}
\newcommand{\ImportTok}[1]{#1}
\newcommand{\InformationTok}[1]{\textcolor[rgb]{0.56,0.35,0.01}{\textbf{\textit{#1}}}}
\newcommand{\KeywordTok}[1]{\textcolor[rgb]{0.13,0.29,0.53}{\textbf{#1}}}
\newcommand{\NormalTok}[1]{#1}
\newcommand{\OperatorTok}[1]{\textcolor[rgb]{0.81,0.36,0.00}{\textbf{#1}}}
\newcommand{\OtherTok}[1]{\textcolor[rgb]{0.56,0.35,0.01}{#1}}
\newcommand{\PreprocessorTok}[1]{\textcolor[rgb]{0.56,0.35,0.01}{\textit{#1}}}
\newcommand{\RegionMarkerTok}[1]{#1}
\newcommand{\SpecialCharTok}[1]{\textcolor[rgb]{0.00,0.00,0.00}{#1}}
\newcommand{\SpecialStringTok}[1]{\textcolor[rgb]{0.31,0.60,0.02}{#1}}
\newcommand{\StringTok}[1]{\textcolor[rgb]{0.31,0.60,0.02}{#1}}
\newcommand{\VariableTok}[1]{\textcolor[rgb]{0.00,0.00,0.00}{#1}}
\newcommand{\VerbatimStringTok}[1]{\textcolor[rgb]{0.31,0.60,0.02}{#1}}
\newcommand{\WarningTok}[1]{\textcolor[rgb]{0.56,0.35,0.01}{\textbf{\textit{#1}}}}
\usepackage{graphicx}
% grffile has become a legacy package: https://ctan.org/pkg/grffile
\IfFileExists{grffile.sty}{%
\usepackage{grffile}
}{}
\makeatletter
\def\maxwidth{\ifdim\Gin@nat@width>\linewidth\linewidth\else\Gin@nat@width\fi}
\def\maxheight{\ifdim\Gin@nat@height>\textheight\textheight\else\Gin@nat@height\fi}
\makeatother
% Scale images if necessary, so that they will not overflow the page
% margins by default, and it is still possible to overwrite the defaults
% using explicit options in \includegraphics[width, height, ...]{}
\setkeys{Gin}{width=\maxwidth,height=\maxheight,keepaspectratio}
\IfFileExists{parskip.sty}{%
\usepackage{parskip}
}{% else
\setlength{\parindent}{0pt}
\setlength{\parskip}{6pt plus 2pt minus 1pt}
}
\setlength{\emergencystretch}{3em}  % prevent overfull lines
\providecommand{\tightlist}{%
  \setlength{\itemsep}{0pt}\setlength{\parskip}{0pt}}
\setcounter{secnumdepth}{0}
% Redefines (sub)paragraphs to behave more like sections
\ifx\paragraph\undefined\else
\let\oldparagraph\paragraph
\renewcommand{\paragraph}[1]{\oldparagraph{#1}\mbox{}}
\fi
\ifx\subparagraph\undefined\else
\let\oldsubparagraph\subparagraph
\renewcommand{\subparagraph}[1]{\oldsubparagraph{#1}\mbox{}}
\fi

%%% Use protect on footnotes to avoid problems with footnotes in titles
\let\rmarkdownfootnote\footnote%
\def\footnote{\protect\rmarkdownfootnote}

%%% Change title format to be more compact
\usepackage{titling}

% Create subtitle command for use in maketitle
\providecommand{\subtitle}[1]{
  \posttitle{
    \begin{center}\large#1\end{center}
    }
}

\setlength{\droptitle}{-2em}

  \title{Causal Inference: HW1}
    \pretitle{\vspace{\droptitle}\centering\huge}
  \posttitle{\par}
    \author{Mahdi Sabbaghi (95109123)}
    \preauthor{\centering\large\emph}
  \postauthor{\par}
    \date{}
    \predate{}\postdate{}
  

\begin{document}
\maketitle

\#Part1: At this part I used rnorm to generate standard normal the
samples. for ploting the conditional distributions I've used three
different x and for every x I generate 1000 samples of y.

\begin{Shaded}
\begin{Highlighting}[]
\NormalTok{x<-}\StringTok{ }\KeywordTok{rnorm}\NormalTok{(}\DecValTok{1000}\NormalTok{, }\DecValTok{0}\NormalTok{, }\DecValTok{1}\NormalTok{)}
\NormalTok{y<-}\StringTok{ }\NormalTok{x}\OperatorTok{^}\DecValTok{3}\OperatorTok{+}\KeywordTok{rnorm}\NormalTok{(}\DecValTok{1000}\NormalTok{, }\DecValTok{0}\NormalTok{ , }\DecValTok{1}\NormalTok{)}
\KeywordTok{options}\NormalTok{(}\DataTypeTok{repr.plot.width=} \DecValTok{3}\NormalTok{, }\DataTypeTok{repr.plot.height=} \DecValTok{3}\NormalTok{)}
\KeywordTok{plot}\NormalTok{(x, y)}
\end{Highlighting}
\end{Shaded}

\includegraphics{HW1_files/figure-latex/unnamed-chunk-1-1.pdf}

\begin{Shaded}
\begin{Highlighting}[]
\NormalTok{x=}\StringTok{ }\DecValTok{0}
\NormalTok{y<-}\StringTok{ }\NormalTok{x}\OperatorTok{^}\DecValTok{3}\OperatorTok{+}\KeywordTok{rnorm}\NormalTok{(}\DecValTok{1000}\NormalTok{, }\DecValTok{0}\NormalTok{ , }\DecValTok{1}\NormalTok{)}
\KeywordTok{hist}\NormalTok{(y, }\DataTypeTok{breaks =}  \DecValTok{50}\NormalTok{, }\DataTypeTok{main=} \StringTok{"P(y|x= 0)"}\NormalTok{)}
\end{Highlighting}
\end{Shaded}

\includegraphics{HW1_files/figure-latex/unnamed-chunk-1-2.pdf}

\begin{Shaded}
\begin{Highlighting}[]
\NormalTok{x=}\StringTok{ }\DecValTok{2}
\NormalTok{y<-}\StringTok{ }\NormalTok{x}\OperatorTok{^}\DecValTok{3}\OperatorTok{+}\KeywordTok{rnorm}\NormalTok{(}\DecValTok{1000}\NormalTok{, }\DecValTok{0}\NormalTok{ , }\DecValTok{1}\NormalTok{)}
\KeywordTok{hist}\NormalTok{(y, }\DataTypeTok{breaks =}  \DecValTok{50}\NormalTok{, }\DataTypeTok{main=} \StringTok{"P(y|x= 2)"}\NormalTok{)}
\end{Highlighting}
\end{Shaded}

\includegraphics{HW1_files/figure-latex/unnamed-chunk-1-3.pdf}

\begin{Shaded}
\begin{Highlighting}[]
\NormalTok{x=}\StringTok{ }\DecValTok{-2}
\NormalTok{y<-}\StringTok{ }\NormalTok{x}\OperatorTok{^}\DecValTok{3}\OperatorTok{+}\KeywordTok{rnorm}\NormalTok{(}\DecValTok{1000}\NormalTok{, }\DecValTok{0}\NormalTok{ , }\DecValTok{1}\NormalTok{)}
\KeywordTok{hist}\NormalTok{(y, }\DataTypeTok{breaks =}  \DecValTok{50}\NormalTok{, }\DataTypeTok{main=} \StringTok{"P(y|x= -2)"}\NormalTok{)}
\end{Highlighting}
\end{Shaded}

\includegraphics{HW1_files/figure-latex/unnamed-chunk-1-4.pdf} for
deriving \(P(X|Y)\) again I used three different values of y=0, 2, -2.
noting that by using Baysian's rule:
\[P(X|Y=y)= \frac{P(Y=y|X)P(X)}{P(Y=y)}\] so using that:

\begin{Shaded}
\begin{Highlighting}[]
\NormalTok{x=}\StringTok{ }\KeywordTok{seq}\NormalTok{(}\OperatorTok{-}\DecValTok{10}\NormalTok{, }\DecValTok{10}\NormalTok{ , }\FloatTok{0.05}\NormalTok{)}
\NormalTok{y=}\DecValTok{0}
\NormalTok{P1=}\StringTok{ }\DecValTok{1}\OperatorTok{/}\DecValTok{2}\OperatorTok{*}\KeywordTok{exp}\NormalTok{(}\OperatorTok{-}\NormalTok{((y}\OperatorTok{-}\StringTok{ }\NormalTok{x}\OperatorTok{^}\DecValTok{3}\NormalTok{)}\OperatorTok{^}\DecValTok{2}\NormalTok{)}\OperatorTok{/}\DecValTok{2}\NormalTok{)}\OperatorTok{*}\KeywordTok{exp}\NormalTok{(}\OperatorTok{-}\NormalTok{(x}\OperatorTok{^}\DecValTok{2}\NormalTok{)}\OperatorTok{/}\DecValTok{2}\NormalTok{)}
\KeywordTok{plot}\NormalTok{(x, P1, }\DataTypeTok{main =} \StringTok{"P(X|Y=0)"}\NormalTok{)}
\end{Highlighting}
\end{Shaded}

\includegraphics{HW1_files/figure-latex/unnamed-chunk-2-1.pdf}

\begin{Shaded}
\begin{Highlighting}[]
\NormalTok{y=}\DecValTok{1}
\NormalTok{P2=}\StringTok{ }\DecValTok{1}\OperatorTok{/}\DecValTok{2}\OperatorTok{*}\KeywordTok{exp}\NormalTok{(}\OperatorTok{-}\NormalTok{((y}\OperatorTok{-}\StringTok{ }\NormalTok{x}\OperatorTok{^}\DecValTok{3}\NormalTok{)}\OperatorTok{^}\DecValTok{2}\NormalTok{)}\OperatorTok{/}\DecValTok{2}\NormalTok{)}\OperatorTok{*}\KeywordTok{exp}\NormalTok{(}\OperatorTok{-}\NormalTok{(x}\OperatorTok{^}\DecValTok{2}\NormalTok{)}\OperatorTok{/}\DecValTok{2}\NormalTok{)}
\KeywordTok{plot}\NormalTok{(x, P2, }\DataTypeTok{main =} \StringTok{"P(X|Y=1)"}\NormalTok{)}
\end{Highlighting}
\end{Shaded}

\includegraphics{HW1_files/figure-latex/unnamed-chunk-2-2.pdf}

\begin{Shaded}
\begin{Highlighting}[]
\NormalTok{y=}\OperatorTok{-}\DecValTok{1}
\NormalTok{P3=}\StringTok{ }\DecValTok{1}\OperatorTok{/}\DecValTok{2}\OperatorTok{*}\KeywordTok{exp}\NormalTok{(}\OperatorTok{-}\NormalTok{((y}\OperatorTok{-}\StringTok{ }\NormalTok{x}\OperatorTok{^}\DecValTok{3}\NormalTok{)}\OperatorTok{^}\DecValTok{2}\NormalTok{)}\OperatorTok{/}\DecValTok{2}\NormalTok{)}\OperatorTok{*}\KeywordTok{exp}\NormalTok{(}\OperatorTok{-}\NormalTok{(x}\OperatorTok{^}\DecValTok{2}\NormalTok{)}\OperatorTok{/}\DecValTok{2}\NormalTok{)}
\KeywordTok{plot}\NormalTok{(x, P3, }\DataTypeTok{main =} \StringTok{"P(X|Y=-1)"}\NormalTok{)}
\end{Highlighting}
\end{Shaded}

\includegraphics{HW1_files/figure-latex/unnamed-chunk-2-3.pdf} as we
know the conditional distribution and intervention on the cause have no
differences. However, inteventing on the effect(y in this problem)
doesn't make any change on X. therefore:
\[P_{X}^{do(Y=y);}(x)= P_{X}(x)\] \#Part1.2: for ``t-distribution'' I've
used function ``rt'' and as before:

\begin{Shaded}
\begin{Highlighting}[]
\NormalTok{x<-}\StringTok{ }\KeywordTok{rnorm}\NormalTok{(}\DecValTok{1000}\NormalTok{, }\DecValTok{0}\NormalTok{, }\DecValTok{1}\NormalTok{)}
\NormalTok{y<-}\StringTok{ }\NormalTok{x}\OperatorTok{^}\DecValTok{3}\OperatorTok{+}\KeywordTok{rt}\NormalTok{(}\DecValTok{1000}\NormalTok{, }\DataTypeTok{df=} \DecValTok{1}\NormalTok{)}
\KeywordTok{plot}\NormalTok{(x, y)}
\end{Highlighting}
\end{Shaded}

\includegraphics{HW1_files/figure-latex/unnamed-chunk-3-1.pdf}

\begin{Shaded}
\begin{Highlighting}[]
\NormalTok{x=}\StringTok{ }\DecValTok{0}
\NormalTok{y<-}\StringTok{ }\NormalTok{x}\OperatorTok{^}\DecValTok{3}\OperatorTok{+}\KeywordTok{rt}\NormalTok{(}\DecValTok{1000}\NormalTok{, }\DataTypeTok{df=} \DecValTok{1}\NormalTok{)}
\KeywordTok{hist}\NormalTok{(y, }\DataTypeTok{breaks =}  \DecValTok{50}\NormalTok{, }\DataTypeTok{main=} \StringTok{"P(y|x= 0)"}\NormalTok{)}
\end{Highlighting}
\end{Shaded}

\includegraphics{HW1_files/figure-latex/unnamed-chunk-3-2.pdf}

\begin{Shaded}
\begin{Highlighting}[]
\NormalTok{x=}\StringTok{ }\DecValTok{2}
\NormalTok{y<-}\StringTok{ }\NormalTok{x}\OperatorTok{^}\DecValTok{3}\OperatorTok{+}\KeywordTok{rt}\NormalTok{(}\DecValTok{1000}\NormalTok{, }\DataTypeTok{df=} \DecValTok{1}\NormalTok{)}
\KeywordTok{hist}\NormalTok{(y, }\DataTypeTok{breaks =}  \DecValTok{50}\NormalTok{, }\DataTypeTok{main=} \StringTok{"P(y|x= 2)"}\NormalTok{)}
\end{Highlighting}
\end{Shaded}

\includegraphics{HW1_files/figure-latex/unnamed-chunk-3-3.pdf}

\begin{Shaded}
\begin{Highlighting}[]
\NormalTok{x=}\StringTok{ }\DecValTok{-2}
\NormalTok{y<-}\StringTok{ }\NormalTok{x}\OperatorTok{^}\DecValTok{3}\OperatorTok{+}\KeywordTok{rt}\NormalTok{(}\DecValTok{1000}\NormalTok{, }\DataTypeTok{df=} \DecValTok{1}\NormalTok{)}
\KeywordTok{hist}\NormalTok{(y, }\DataTypeTok{breaks =}  \DecValTok{50}\NormalTok{, }\DataTypeTok{main=} \StringTok{"P(y|x= -2)"}\NormalTok{)}
\end{Highlighting}
\end{Shaded}

\includegraphics{HW1_files/figure-latex/unnamed-chunk-3-4.pdf} the
result is odd because the tail of t-distribution with df=1 does not go
to zero as we want to. for the next part:

\begin{Shaded}
\begin{Highlighting}[]
\NormalTok{x=}\StringTok{ }\KeywordTok{seq}\NormalTok{(}\OperatorTok{-}\DecValTok{10}\NormalTok{, }\DecValTok{10}\NormalTok{ , }\FloatTok{0.05}\NormalTok{)}
\NormalTok{y=}\DecValTok{0}
\NormalTok{P1=}\StringTok{ }\KeywordTok{dt}\NormalTok{(y}\OperatorTok{-}\StringTok{ }\NormalTok{x}\OperatorTok{^}\DecValTok{3}\NormalTok{, }\DataTypeTok{df=} \DecValTok{1}\NormalTok{)}\OperatorTok{*}\KeywordTok{dnorm}\NormalTok{(x)}
\KeywordTok{plot}\NormalTok{(x, P1, }\DataTypeTok{main =} \StringTok{"P(X|Y=0)"}\NormalTok{)}
\end{Highlighting}
\end{Shaded}

\includegraphics{HW1_files/figure-latex/unnamed-chunk-4-1.pdf}

\begin{Shaded}
\begin{Highlighting}[]
\NormalTok{y=}\DecValTok{1}
\NormalTok{P2=}\StringTok{ }\KeywordTok{dt}\NormalTok{(y}\OperatorTok{-}\StringTok{ }\NormalTok{x}\OperatorTok{^}\DecValTok{3}\NormalTok{, }\DataTypeTok{df=} \DecValTok{1}\NormalTok{)}\OperatorTok{*}\KeywordTok{dnorm}\NormalTok{(x)}
\KeywordTok{plot}\NormalTok{(x, P2, }\DataTypeTok{main =} \StringTok{"P(X|Y=1)"}\NormalTok{)}
\end{Highlighting}
\end{Shaded}

\includegraphics{HW1_files/figure-latex/unnamed-chunk-4-2.pdf}

\begin{Shaded}
\begin{Highlighting}[]
\NormalTok{y=}\OperatorTok{-}\DecValTok{1}
\NormalTok{P3=}\StringTok{ }\KeywordTok{dt}\NormalTok{(y}\OperatorTok{-}\StringTok{ }\NormalTok{x}\OperatorTok{^}\DecValTok{3}\NormalTok{, }\DataTypeTok{df=} \DecValTok{1}\NormalTok{)}\OperatorTok{*}\KeywordTok{dnorm}\NormalTok{(x)}
\KeywordTok{plot}\NormalTok{(x, P3, }\DataTypeTok{main =} \StringTok{"P(X|Y=-1)"}\NormalTok{)}
\end{Highlighting}
\end{Shaded}

\includegraphics{HW1_files/figure-latex/unnamed-chunk-4-3.pdf} \#Part
1.3: this is just as before except df= 5. So I just note the results:

\begin{Shaded}
\begin{Highlighting}[]
\NormalTok{x<-}\StringTok{ }\KeywordTok{rnorm}\NormalTok{(}\DecValTok{1000}\NormalTok{, }\DecValTok{0}\NormalTok{, }\DecValTok{1}\NormalTok{)}
\NormalTok{y<-}\StringTok{ }\NormalTok{x}\OperatorTok{^}\DecValTok{3}\OperatorTok{+}\KeywordTok{rt}\NormalTok{(}\DecValTok{1000}\NormalTok{, }\DataTypeTok{df=} \DecValTok{5}\NormalTok{)}
\KeywordTok{plot}\NormalTok{(x, y)}
\end{Highlighting}
\end{Shaded}

\includegraphics{HW1_files/figure-latex/unnamed-chunk-5-1.pdf}

\begin{Shaded}
\begin{Highlighting}[]
\NormalTok{x=}\StringTok{ }\DecValTok{0}
\NormalTok{y<-}\StringTok{ }\NormalTok{x}\OperatorTok{^}\DecValTok{3}\OperatorTok{+}\KeywordTok{rt}\NormalTok{(}\DecValTok{1000}\NormalTok{, }\DataTypeTok{df=} \DecValTok{5}\NormalTok{)}
\KeywordTok{hist}\NormalTok{(y, }\DataTypeTok{breaks =}  \DecValTok{50}\NormalTok{, }\DataTypeTok{main=} \StringTok{"P(y|x= 0)"}\NormalTok{)}
\end{Highlighting}
\end{Shaded}

\includegraphics{HW1_files/figure-latex/unnamed-chunk-5-2.pdf}

\begin{Shaded}
\begin{Highlighting}[]
\NormalTok{x=}\StringTok{ }\DecValTok{2}
\NormalTok{y<-}\StringTok{ }\NormalTok{x}\OperatorTok{^}\DecValTok{3}\OperatorTok{+}\KeywordTok{rt}\NormalTok{(}\DecValTok{1000}\NormalTok{, }\DataTypeTok{df=} \DecValTok{5}\NormalTok{)}
\KeywordTok{hist}\NormalTok{(y, }\DataTypeTok{breaks =}  \DecValTok{50}\NormalTok{, }\DataTypeTok{main=} \StringTok{"P(y|x= 2)"}\NormalTok{)}
\end{Highlighting}
\end{Shaded}

\includegraphics{HW1_files/figure-latex/unnamed-chunk-5-3.pdf}

\begin{Shaded}
\begin{Highlighting}[]
\NormalTok{x=}\StringTok{ }\DecValTok{-2}
\NormalTok{y<-}\StringTok{ }\NormalTok{x}\OperatorTok{^}\DecValTok{3}\OperatorTok{+}\KeywordTok{rt}\NormalTok{(}\DecValTok{1000}\NormalTok{, }\DataTypeTok{df=} \DecValTok{5}\NormalTok{)}
\KeywordTok{hist}\NormalTok{(y, }\DataTypeTok{breaks =}  \DecValTok{50}\NormalTok{, }\DataTypeTok{main=} \StringTok{"P(y|x= -2)"}\NormalTok{)}
\end{Highlighting}
\end{Shaded}

\includegraphics{HW1_files/figure-latex/unnamed-chunk-5-4.pdf} As
figures depict above, It's far more concentrated because df= 5. In Part
2 some examples was too far from the mean point.

\begin{Shaded}
\begin{Highlighting}[]
\NormalTok{x=}\StringTok{ }\KeywordTok{seq}\NormalTok{(}\OperatorTok{-}\DecValTok{10}\NormalTok{, }\DecValTok{10}\NormalTok{ , }\FloatTok{0.05}\NormalTok{)}
\NormalTok{y=}\DecValTok{0}
\NormalTok{P1=}\StringTok{ }\KeywordTok{dt}\NormalTok{(y}\OperatorTok{-}\StringTok{ }\NormalTok{x}\OperatorTok{^}\DecValTok{3}\NormalTok{, }\DataTypeTok{df=} \DecValTok{5}\NormalTok{)}\OperatorTok{*}\KeywordTok{dnorm}\NormalTok{(x)}
\KeywordTok{plot}\NormalTok{(x, P1, }\DataTypeTok{main =} \StringTok{"P(X|Y=0)"}\NormalTok{)}
\end{Highlighting}
\end{Shaded}

\includegraphics{HW1_files/figure-latex/unnamed-chunk-6-1.pdf}

\begin{Shaded}
\begin{Highlighting}[]
\NormalTok{y=}\DecValTok{1}
\NormalTok{P2=}\StringTok{ }\KeywordTok{dt}\NormalTok{(y}\OperatorTok{-}\StringTok{ }\NormalTok{x}\OperatorTok{^}\DecValTok{3}\NormalTok{, }\DataTypeTok{df=} \DecValTok{5}\NormalTok{)}\OperatorTok{*}\KeywordTok{dnorm}\NormalTok{(x)}
\KeywordTok{plot}\NormalTok{(x, P2, }\DataTypeTok{main =} \StringTok{"P(X|Y=1)"}\NormalTok{)}
\end{Highlighting}
\end{Shaded}

\includegraphics{HW1_files/figure-latex/unnamed-chunk-6-2.pdf}

\begin{Shaded}
\begin{Highlighting}[]
\NormalTok{y=}\OperatorTok{-}\DecValTok{1}
\NormalTok{P3=}\StringTok{ }\KeywordTok{dt}\NormalTok{(y}\OperatorTok{-}\StringTok{ }\NormalTok{x}\OperatorTok{^}\DecValTok{3}\NormalTok{, }\DataTypeTok{df=} \DecValTok{5}\NormalTok{)}\OperatorTok{*}\KeywordTok{dnorm}\NormalTok{(x)}
\KeywordTok{plot}\NormalTok{(x, P3, }\DataTypeTok{main =} \StringTok{"P(X|Y=-1)"}\NormalTok{)}
\end{Highlighting}
\end{Shaded}

\includegraphics{HW1_files/figure-latex/unnamed-chunk-6-3.pdf} the
difference between part 2 and 3 is that in part 3 two maximum points are
closer.

\#Part 1.4: Again It's like part 2 but df= 20!

\begin{Shaded}
\begin{Highlighting}[]
\NormalTok{x<-}\StringTok{ }\KeywordTok{rnorm}\NormalTok{(}\DecValTok{1000}\NormalTok{, }\DecValTok{0}\NormalTok{, }\DecValTok{1}\NormalTok{)}
\NormalTok{y<-}\StringTok{ }\NormalTok{x}\OperatorTok{^}\DecValTok{3}\OperatorTok{+}\KeywordTok{rt}\NormalTok{(}\DecValTok{1000}\NormalTok{, }\DataTypeTok{df=} \DecValTok{20}\NormalTok{)}
\KeywordTok{plot}\NormalTok{(x, y)}
\end{Highlighting}
\end{Shaded}

\includegraphics{HW1_files/figure-latex/unnamed-chunk-7-1.pdf}

\begin{Shaded}
\begin{Highlighting}[]
\NormalTok{x=}\StringTok{ }\DecValTok{0}
\NormalTok{y<-}\StringTok{ }\NormalTok{x}\OperatorTok{^}\DecValTok{3}\OperatorTok{+}\KeywordTok{rt}\NormalTok{(}\DecValTok{1000}\NormalTok{, }\DataTypeTok{df=} \DecValTok{20}\NormalTok{)}
\KeywordTok{hist}\NormalTok{(y, }\DataTypeTok{breaks =}  \DecValTok{50}\NormalTok{, }\DataTypeTok{main=} \StringTok{"P(y|x= 0)"}\NormalTok{)}
\end{Highlighting}
\end{Shaded}

\includegraphics{HW1_files/figure-latex/unnamed-chunk-7-2.pdf}

\begin{Shaded}
\begin{Highlighting}[]
\NormalTok{x=}\StringTok{ }\DecValTok{2}
\NormalTok{y<-}\StringTok{ }\NormalTok{x}\OperatorTok{^}\DecValTok{3}\OperatorTok{+}\KeywordTok{rt}\NormalTok{(}\DecValTok{1000}\NormalTok{, }\DataTypeTok{df=} \DecValTok{20}\NormalTok{)}
\KeywordTok{hist}\NormalTok{(y, }\DataTypeTok{breaks =}  \DecValTok{50}\NormalTok{, }\DataTypeTok{main=} \StringTok{"P(y|x= 2)"}\NormalTok{)}
\end{Highlighting}
\end{Shaded}

\includegraphics{HW1_files/figure-latex/unnamed-chunk-7-3.pdf}

\begin{Shaded}
\begin{Highlighting}[]
\NormalTok{x=}\StringTok{ }\DecValTok{-2}
\NormalTok{y<-}\StringTok{ }\NormalTok{x}\OperatorTok{^}\DecValTok{3}\OperatorTok{+}\KeywordTok{rt}\NormalTok{(}\DecValTok{1000}\NormalTok{, }\DataTypeTok{df=} \DecValTok{20}\NormalTok{)}
\KeywordTok{hist}\NormalTok{(y, }\DataTypeTok{breaks =}  \DecValTok{50}\NormalTok{, }\DataTypeTok{main=} \StringTok{"P(y|x= -2)"}\NormalTok{)}
\end{Highlighting}
\end{Shaded}

\includegraphics{HW1_files/figure-latex/unnamed-chunk-7-4.pdf} They are
more concentrated than part 2. and for \(P(Y|X)\):

\begin{Shaded}
\begin{Highlighting}[]
\NormalTok{x=}\StringTok{ }\KeywordTok{seq}\NormalTok{(}\OperatorTok{-}\DecValTok{10}\NormalTok{, }\DecValTok{10}\NormalTok{ , }\FloatTok{0.05}\NormalTok{)}
\NormalTok{y=}\DecValTok{0}
\NormalTok{P1=}\StringTok{ }\KeywordTok{dt}\NormalTok{(y}\OperatorTok{-}\StringTok{ }\NormalTok{x}\OperatorTok{^}\DecValTok{3}\NormalTok{, }\DataTypeTok{df=} \DecValTok{20}\NormalTok{)}\OperatorTok{*}\KeywordTok{dnorm}\NormalTok{(x)}
\KeywordTok{plot}\NormalTok{(x, P1, }\DataTypeTok{main =} \StringTok{"P(X|Y=0)"}\NormalTok{)}
\end{Highlighting}
\end{Shaded}

\includegraphics{HW1_files/figure-latex/unnamed-chunk-8-1.pdf}

\begin{Shaded}
\begin{Highlighting}[]
\NormalTok{y=}\DecValTok{1}
\NormalTok{P2=}\StringTok{ }\KeywordTok{dt}\NormalTok{(y}\OperatorTok{-}\StringTok{ }\NormalTok{x}\OperatorTok{^}\DecValTok{3}\NormalTok{, }\DataTypeTok{df=} \DecValTok{20}\NormalTok{)}\OperatorTok{*}\KeywordTok{dnorm}\NormalTok{(x)}
\KeywordTok{plot}\NormalTok{(x, P2, }\DataTypeTok{main =} \StringTok{"P(X|Y=1)"}\NormalTok{)}
\end{Highlighting}
\end{Shaded}

\includegraphics{HW1_files/figure-latex/unnamed-chunk-8-2.pdf}

\begin{Shaded}
\begin{Highlighting}[]
\NormalTok{y=}\OperatorTok{-}\DecValTok{1}
\NormalTok{P3=}\StringTok{ }\KeywordTok{dt}\NormalTok{(y}\OperatorTok{-}\StringTok{ }\NormalTok{x}\OperatorTok{^}\DecValTok{3}\NormalTok{, }\DataTypeTok{df=} \DecValTok{20}\NormalTok{)}\OperatorTok{*}\KeywordTok{dnorm}\NormalTok{(x)}
\KeywordTok{plot}\NormalTok{(x, P3, }\DataTypeTok{main =} \StringTok{"P(X|Y=-1)"}\NormalTok{)}
\end{Highlighting}
\end{Shaded}

\includegraphics{HW1_files/figure-latex/unnamed-chunk-8-3.pdf} note: for
part 2, 3 and 4 I deduced as Part 1 that:
\[P_{Y}^{do(X);}(y)= P_{Y|X}(y|x)\] and: \[P_{X}^{do(Y);}(x)= P(x)\] And
for the direct of the distribution, observe that for all four
distribtion \(P(X|Y=0)\) is much more wider than, sat \(P(X|Y= 20)\).
However this fact is less conspicuous for \(P(Y|X)\) and we can conclude
that for all four distributions that \(x \to y\).


\end{document}
